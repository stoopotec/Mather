\documentclass{article}


\usepackage{fontspec}

\setmainfont{Times New Roman}

\newfontfamily\cyrillicfonttt{Hack}

\newfontfamily\monofont{Hack}


\usepackage{needspace}



\usepackage{polyglossia}
\setmainlanguage{russian}


\usepackage{amsmath, amssymb, amsfonts}
\usepackage{hyperref}
\usepackage{tikz, float, pgfplots}

\usepackage{svg}


\usepackage[margin=3cm]{geometry}




\usepackage{listings}
\usepackage{color}


\lstset{frame=none,
  aboveskip=3mm,
  belowskip=3mm,
  showstringspaces=false,
  columns=flexible,
  basicstyle={\small\ttfamily},
  numbers=none,
  numberstyle=\tiny\color{gray},
  keywordstyle=\color{blue},
  commentstyle=\color{dkgreen},
  stringstyle=\color{mauve},
  breaklines=true,
  breakatwhitespace=true,
  tabsize=3
}



\usepackage{graphicx}











\begin{document}


    \begin{titlepage}

        \Large

        \begin{center}
            
            
            МИНИСТЕРСТВО ОБРАЗОВАНИЯ РЕСПУБЛИКИ БЕЛАРУСЬ

            Учреждение образования «БЕЛОРУССКИЙ ГОСУДАРСТВЕННЫЙ 

            ТЕХНОЛОГИЧЕСКИЙ УНИВЕРСИТЕТ»

        \end{center}
        
        \vspace{1cm}
        
        \begin{tabular}{ll}
            Факультет & Информационных технологий \\
            Кафедра & Информационных систем и технологий \\
            Специальность & 6-05-0611-01 «Информационные системы и технологии» \\
        \end{tabular}

        \vspace{1cm}
        
        \begin{center}
            {\bfseries ПОЯСНИТЕЛЬНАЯ ЗАПИСКА КУРСОВОГО ПРОЕКТА}
        \end{center}

        \vspace{1cm}
        
        по дисциплине «Компьютерные языки разметки»

        Тема: Веб-сайт «Калькулятор простых алгебраических выражений»
        
        \vspace{2cm}
        
        \begin{minipage}{0.5\textwidth}
        \underline{\hspace{2cm}} \\
        \footnotesize{Введите текст здесь}
        \end{minipage}

%         Исполнитель
% студент 1 курса 1 группы		______________	Д. С. Жук
% подпись, дата
% Руководитель
% 	           ассистент                           	______________		Д. В. Сазонова
% должность, учен. степень, ученое звание	подпись, дата

        \begin{flushright}
            Выполнил:
            
            Крупкевич Иван Андреевич

            БГТУ ИСИТ 1--1
        \end{flushright}

        \vspace{2cm}

        \begin{flushright}
            Проверил:
            
        \end{flushright}
        
        
        \vspace{\fill}
        
        \begin{center}
            Минск \the\year
        \end{center}

    \end{titlepage}


    
    \renewcommand{\contentsname}{Содержание}
    
    \tableofcontents
    \newpage
    



    \section{Постановка задачи}

    \begin{enumerate}
        \item Разработать веб-сайт с интерфейсом для ввода выражения
        и отображения результатов его вычисления.
        
        \item Написать сервер на языке программирования C, 
        который будет принимать запросы от веб-сайта, обрабатывать 
        их и возвращать результаты вычислений.
        
        \item Реализовать алгоритмы для обработки простых алгебраических выражений, 
        включая операции сложения, вычитания, умножения и деления.
    
        \item Обеспечить корректную работу веб-сайта и сервера, 
        а также обработку возможных ошибок при вводе данных пользователем.
    \end{enumerate}
    
    Сделать выводы о проделанной работе.




    \subsection{Обзор аналогичных решений}



    \subsubsection{Wolframalpha} https://wolframalpha.com

    WolframAlpha представляет собой мощный вычислительный инструмент, способный решать разнообразные математические задачи, предоставляя детальные ответы и графики.
    
    \subsubsection{Photomath} https://photomath.com/

    В отличие от WolframAlpha, приложение Photomath предлагает уникальную функцию распознавания рукописных математических выражений с помощью камеры смартфона. Этот инновационный подход позволяет пользователям с легкостью вводить математические задачи, просто сфотографировав их, и получать мгновенные ответы. Photomath фокусируется на удобстве использования и доступности, делая процесс решения математических задач более интерактивным и интуитивным для пользователей всех уровней.


    \subsection{Техническое задание}
    
    \begin{enumerate}
        \item \textbf{Информация о проекте:}
        \begin{itemize}
            \item Количество веб-страниц: 3
            \item Страницы и их содержание:
            \begin{itemize}
                \item Первая страница: содержит большую кнопку "Dive In", которая переносит пользователя в удивительный мир математики.
                \item Вторая страница: список теорем с возможностью просмотра их доказательств.
                \item Третья страница: позволяет пользователю вычислить значение математического выражения.
            \end{itemize}
        \end{itemize}
        
        \item \textbf{Формирование требований к программному продукту:}
        \begin{itemize}
            \item Разработать веб-сайт с интуитивным интерфейсом и привлекательным дизайном.
            \item Обеспечить быструю и удобную навигацию между страницами.
            \item Создать страницу с перечнем теорем и возможностью просмотра их доказательств.
            \item Разработать страницу для вычисления значений математических выражений с удобным интерфейсом для ввода данных.
        \end{itemize}
        
        \item \textbf{Формулировка задач программного продукта:}
        \begin{itemize}
            \item Создать страницу с перечнем теорем и возможностью просмотра их доказательств.
            \item Разработать функционал для вычисления значений математических выражений на отдельной странице.
            \item Обеспечить корректную работу всех элементов интерфейса и функций веб-сайта.
            \item Предоставить пользователю удобный и интерактивный опыт взаимодействия с математическими данными и вычислениями.
        \end{itemize}
    \end{enumerate}
    
    
    \subsection{Выбор средств реализации программного продукта}

    При принятии решения о выборе средств для реализации программного продукта, я решил использовать среду разработки Visual Studio Code (VS Code). VS Code предоставляет удобную и гибкую среду для написания кода, обладает широким набором расширений и инструментов, что позволяет увеличить производительность и удобство разработки. Благодаря интеграции с различными языками программирования, отладчиками и системами контроля версий, VS Code обеспечивает эффективное взаимодействие с кодом и управление проектом. В результате использования VS Code я смог повысить эффективность разработки и улучшить качество программного продукта.

    Дополнительно, для отображения математических формул на веб-сайте, я использовал библиотеку KaTeX. KaTeX обеспечивает быструю и качественную отрисовку математических выражений непосредственно в браузере, что позволяет представлять алгебраические формулы на веб-странице в удобном и профессиональном виде. Благодаря интеграции KaTeX в мой веб-сайт, пользователи могут легко вводить и просматривать математические выражения, что повышает удобство использования калькулятора и обеспечивает более качественный пользовательский опыт.

    \subsubsection{Вывод}

    Определен основной функционал программного продукта, включающий создание веб-сайта для математических вычислений с функцией перехода в интерактивный раздел, просмотром теорем и их доказательством, а также возможностью вычисления математических выражений. Для реализации проекта планируется использовать технологии KaTeX для отображения математических формул, язык программирования C для серверной части, а также HTML, CSS и JavaScript для создания пользовательского интерфейса веб-сайта. Этот стек технологий обеспечит удобство использования и функциональность программного продукта.
    




    
    \section{Проектирование страниц веб-сайта}


    \subsection{Выбор способа верстки}

    При выборе способа верстки для моего проекта было решено использовать HTML и CSS в сочетании с библиотекой KaTeX. HTML (HyperText Markup Language) является основным языком разметки веб-страниц, который позволяет структурировать контент и создавать элементы интерфейса. CSS (Cascading Style Sheets) используется для стилизации веб-страниц, что позволяет задавать внешний вид элементов и макет страницы.

    KaTeX представляет собой быструю и легковесную библиотеку для отрисовки математических выражений в формате TeX на веб-страницах. Благодаря своей эффективности и производительности, KaTeX обеспечивает быструю загрузку и отображение сложных математических формул без необходимости использования серверного рендеринга.

    Использование HTML, CSS и библиотеки KaTeX в проекте позволило создать удобный и функциональный веб-интерфейс для взаимодействия с математическими данными, обеспечивая пользователю понятное и привлекательное представление информации.


    \subsection{Выбор стилевого оформления}

    При выборе стилевого оформления для моего проекта было принято использовать темную тему с белым текстом и минималистичным дизайном. Темный стиль интерфейса обеспечивает комфортное восприятие контента и снижает нагрузку на глаза пользователей, особенно при длительном использовании в условиях недостаточного освещения. Белый текст на темном фоне создает контрастный эффект, делая информацию более читаемой и выделяя ключевые элементы интерфейса.

    Минималистичный дизайн был выбран для обеспечения простоты и ясности восприятия пользователем. Упрощенный стиль позволяет сосредоточить внимание на основном контенте и функционале веб-сайта, не отвлекая пользователя лишними деталями. Минимализм способствует лаконичности интерфейса, делая его более современным и эстетичным.


    \subsection{Выбор шрифтового оформления}

    Для шрифтового оформления я выбрал обычные шрифты. Основная цель была сделать текст читаемым и легко воспринимаемым. Обычные шрифты подходят для широкого круга пользователей и делают дизайн простым и доступным. Этот выбор помогает пользователям сосредоточиться на информации на сайте без лишних отвлечений.


    \subsection{Разработка логотипа}

    \begin{figure}[h]
        \centering
        \includesvg[scale=3]{logo}
        \caption{Логотип}
    \end{figure}

    При разработке логотипа была использована буква "M", стилизованная в виде числовых множеств, что символизирует связь с математикой. Название "Mather" происходит от слова "Math" - математика, что отражает основную тематику проекта.


    \subsection{Разработка пользовательских элементов}

    \subsubsection{toggle switch}
    Разработка этого элемента интерфейса включала в себя создание ползунка, который плавно перемещается между двумя состояниями, отображая выбор пользователя. Я также добавил анимацию и эффекты перехода, чтобы сделать взаимодействие с переключателем более приятным и интуитивно понятным.


    \subsection{Разработка спецэффектов}

    Для создания красивой светящейся кнопки я использовал псевдоэлемент \texttt{::before}, который добавляет дополнительный слой перед содержимым кнопки. Я задал градиентный фон с помощью \texttt{linear-gradient}, установил анимацию для изменения позиции фона, добавил размытие с помощью \texttt{filter: blur()}, и настроил переход при наведении на кнопку для изменения размытия. Это создает эффект светящейся кнопки с плавными анимациями и стилизацией.


    \subsection{Выводы}

    \begin{enumerate}
        \item \textbf{Важность пользовательского опыта}: Проектирование страниц веб-сайта играет ключевую роль в создании удобного и привлекательного пользовательского опыта. Грамотно спроектированные страницы способствуют легкому взаимодействию пользователя с сайтом и повышают его удовлетворенность.
        
        \item \textbf{Согласованность дизайна}: Важно обеспечить согласованность дизайна на всех страницах веб-сайта. Это помогает создать цельное визуальное впечатление и узнаваемый стиль, что способствует узнаваемости бренда.
        
        \item \textbf{Адаптивность и отзывчивость}: С учетом разнообразия устройств, на которых пользователи могут просматривать сайт, важно обеспечить адаптивность и отзывчивость дизайна. Это позволит сайту корректно отображаться на различных устройствах и улучшит общий пользовательский опыт.
        
        \item \textbf{Удобство навигации}: Хорошо спроектированная навигация на страницах веб-сайта помогает пользователям быстро находить нужную информацию. Четкая структура и интуитивно понятные элементы навигации улучшают пользовательскую интеракцию.
        
        \item \textbf{Тестирование и оптимизация}: Важно проводить тестирование дизайна страниц веб-сайта с целью выявления возможных проблем и улучшения пользовательского опыта. Оптимизация дизайна на основе обратной связи пользователей поможет создать более эффективный и привлекательный сайт.
    \end{enumerate}

    В целом, проектирование страниц веб-сайта играет важную роль в создании успешного онлайн присутствия. Учитывая вышеперечисленные аспекты, можно создать сайт, который не только привлечет пользователей, но и обеспечит им приятный и удобный пользовательский опыт.






    \section{Реализация структуры веб-сайта}

    Представить листинги структуры веб-сайта. Обосновать выбор элементов в коде. 
    Если это большой важный кусок кода, то покажите его в приложение в виде листинга, но укажите ссылку на приложение. Например, скрипт для создания таблиц представлен в приложении Б. В листингах интервалов после абзацев быть не должно.
    \subsection{Структура HTML-документа}
    \subsection{Добавление таблиц стилей CSS}
    \subsection{Использование стандартов XML (SVG)}
    \subsection{Управление элементами DOM}
    \subsection{Выводы}





    \section{Тестирование веб-сайта}

    4.1. Адаптивный дизайн веб-сайта
    Представить как осуществляться будет адаптивность вашего сайта под разные устройста в различных браузерах
    4.2. Кроссбраузерность веб-сайта
    В кроссбраузерное тестирование входит общий вид вашего проекта в других браузерах (сохранились ли шрифты, не съехала ли анимация и так далее)
    4.3. Руководство пользователя
    Описание того, как пользователю пользоваться вашим сайтом со скриншотами, а также как он может взаимодействовать с основными ключевыми функциями вашего продукта.
    4.4. Выводы
    Заключение (подытожить что было сделано вами, что было вами использовано дополнительно не встречающееся в курсе, ссылка на репозитарий Github с веб-сайтом курсового проекта)
    Список использованных источников.
    Приложение (полный исходный текст программы разработанного приложения с подробными комментариями в виде листинга).
    Приложение А Прототипы веб-страниц
    Приложение Б Макет структуры веб-сайта
    Приложение В Листинг НТML-документа
    Приложение Г Листинг SCSS и CSS
    Приложение Д Листинг XML-файлов
    Приложение Е Листинг SVG
    Приложение Ж Листинг JavaScript

    

\end{document}